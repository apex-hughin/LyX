%% LyX trick_preamble_code_into_believing_that_this_was_created_by_lyx created this file.  For more info, see http://www.lyx.org/.
%% Do not edit unless you really know what you are doing.
\documentclass[b4paper,twoside, twocolumn,12pt]{article}

\usepackage[T1]{fontenc}
\usepackage[latin9]{inputenc}

\usepackage{geometry}
\geometry{verbose,landscape,b4paper,tmargin=1cm,bmargin=2cm,lmargin=3cm,rmargin=4cm,headheight=6cm,headsep=5cm,footskip=7cm,columnsep=8cm}

\usepackage{amsmath}
\usepackage{color}
\definecolor{page_backgroundcolor}{rgb}{1, 0.3359375, 0}
\pagecolor{page_backgroundcolor}
\definecolor{document_fontcolor}{rgb}{0.66796875, 1, 0}
\color{document_fontcolor}
\definecolor{note_fontcolor}{rgb}{0, 0, 1}
\definecolor{shadecolor}{rgb}{1, 1, 0}
\usepackage{framed}
\usepackage{calc}
\usepackage{fancybox}

\setlength{\parskip}{3mm}
\setlength{\parindent}{0sp}
\usepackage{setspace}
\setstretch{1.2}

%%%%%%%%%%%%%%%%%%%%%%%%%%%%%% LyX specific LaTeX commands.

%% The greyedout annotation environment
\newenvironment{lyxgreyedout}{\textcolor{note_fontcolor}\bgroup}{\egroup}

%%%%%%%%%%%%%%%%%%%%%%%%%%%%%% User specified LaTeX commands.

\setlength{\fboxsep}{3mm}

\definecolor{darkgreen}{cmyk}{0.5, 0, 1, 0.5}

\usepackage{ifpdf} % part of the hyperref bundle
\ifpdf % if pdflatex is used

 % set fonts for nicer pdf view
 \IfFileExists{lmodern.sty}{\usepackage{lmodern}}{}

 % link all cross references and URLs in pdf output
 \usepackage[colorlinks=true, bookmarks, bookmarksnumbered,
  bookmarksopen, bookmarksopenlevel=2,
  linkcolor=black, citecolor=black, urlcolor=blue, filecolor=blue,
  pdfpagelayout=OneColumn, pdfnewwindow=true,
  pdfstartview=XYZ, plainpages=false, pdfpagelabels,
  pdfauthor={LyX Team}, pdftex,
  pdftitle={tex2lyx testcase},pdfsubject={tex2lyx},
  pdfkeywords={LyX, tex2lyx}]{hyperref}
 
\else % if dvi or ps is produced

 % link all cross references and URLs in dvi output
 \usepackage[ps2pdf]{hyperref}

\fi % end if pdflatex is used


\begin{document}

\tableofcontents

\section{Boxes}

\subsection{Frameless Boxes}

blabla \mbox{mbox} blabla

blabla \makebox{makebox 1} blabla

blabla \makebox[3cm]{makebox 2} blabla

blabla \makebox[3cm][l]{makebox 3} blabla

\begin{figure}[ht]
  \centering
  \setlength{\unitlength}{.2in}

\begin{picture}(8,6)
\put(0,0){\makebox(0,0)[tr]{AAA}}
\put(8,0){\makebox(0,0){BBB}}
\put(1,0){\line(1,0){6}}   
\end{picture}

\end{figure}

\raisebox {8.5mm}{test}\raisebox {-6.5mm}{tset}

\subsection{Framed Boxes}

blabla \begin{framed}framed\end{framed} blabla

blabla \begin{framed}\begin{framed}nested framed\end{framed}\end{framed} blabla

blabla \fbox{fbox} blabla

blabla \framebox{framebox 1} blabla

blabla \framebox[3cm]{framebox 2} blabla

blabla \framebox[3cm][l]{framebox 3} blabla

Dies ist ein Beispieltext. %
\framebox{%
\begin{minipage}[c][1\totalheight][s]{0.2\columnwidth}%
 \begin{center}
  Der Boxinhalt
 \par\end{center}

 \begin{center}
  ist �ber die
 \par\end{center}

 \begin{center}
  Boxh�he
 \par\end{center}

 \begin{center}
  gleichm��ig verteilt.
 \par\end{center}%
\end{minipage}}
Dies ist ein Beispieltext.

blabla \ovalbox{ovalbox} blabla

blabla \Ovalbox{Ovalbox} blabla

blabla \shadowbox{shadowbox} blabla

blabla \doublebox{doublebox} blabla

$\boxed{\int A=B}$

\subsection{LyX Boxes}

\begin{lyxgreyedout}
\textbf{Note:} Not all DVI-viewers are able to display rotations.
\end{lyxgreyedout}

\begin{shaded}%
Shaded background box\end{shaded}

\begin{minipage}[c]{1\columnwidth}%
\begin{shaded}%
Shaded background box, with inner minipage\end{shaded}%
\end{minipage}

\section{Colors}

\subsection{Predefined Colors}

test \textcolor{blue}{blue} test \textcolor{red}{red red red} test \textcolor{green}{bla}\textcolor{magenta}{blub}
test
\textcolor{green}{c}%
\textcolor{red}{o}%
\textcolor{blue}{l}%
\textcolor{green}{o}%
\textcolor{red}{r}

\subsection{Custom Colors}

test \textcolor{darkgreen}{dark green} test

\definecolor{violet}{rgb}{0.5, 0, 1}

test \textcolor{violet}{violet} test

\section{Font sizes}

\Huge Huge text

\huge huge text

\LARGE LARGE text

\Large Large text

\large large text

\normalsize normalsize text

\small small text

\footnotesize footnotesize text

\scriptsize scriptsize text

\tiny tiny text

\Huge Huge text \huge huge text \LARGE LARGE text \Large Large text
\large large text \normalsize normalsize text \small small text
\footnotesize footnotesize text \scriptsize scriptsize text \tiny tiny text

\normalsize Font size switches don't affect section headings!

\scriptsize bla blub

\section{Font size dummy 1}

bla blub

bla blub

\section*{Font size dummy 2}

bla blub
\normalsize


\section{Strikeout, underlined etc.}

Emphasized: \emph{test}

Underbar: \uline{test}

Double underbar: \uuline{test}

Wavy underbar: \uwave{test}

Strike out: \sout{test}

Noun: \noun{test}

Underbar, ephasized, stikreout: \emph{\uline{\sout{test}}}


\section{Paragraph spacing}

bla

\begin{singlespace}
singlespace single singlespace single singlespace single singlespace single
singlespace single singlespace single singlespace single singlespace single
\end{singlespace}

\begin{onehalfspace}
onehalfspace 1 onehalfspace 1 onehalfspace 1 onehalfspace 1 onehalfspace 1
onehalfspace 1 onehalfspace 1 onehalfspace 1 onehalfspace 1 onehalfspace 1

onehalfspace 2 onehalfspace 2 onehalfspace 2 onehalfspace 2 onehalfspace 2
onehalfspace 2 onehalfspace 2 onehalfspace 2 onehalfspace 2 onehalfspace 2
\end{onehalfspace}

blub

\begin{onehalfspace}
onehalfspace single onehalfspace single onehalfspace single onehalfspace single
onehalfspace single onehalfspace single onehalfspace single onehalfspace single
\end{onehalfspace}

blablub

\begin{doublespace}
doublespace single doublespace single doublespace single doublespace single
doublespace single doublespace single doublespace single doublespace single
\end{doublespace}

\begin{spacing}{1.2}
1.2 spacing single 1.2 spacing single 1.2 spacing single 1.2 spacing single
1.2 spacing single 1.2 spacing single 1.2 spacing single 1.2 spacing single
\end{spacing}

\section{Paragraph alignment}

bla

\begin{center}
center single center single center single center single center single
center single center single center single center single center single
\end{center}

blabla

\begin{flushleft}
flushleft 1 flushleft 1 flushleft 1 flushleft 1 flushleft 1 flushleft 1
flushleft 1 flushleft 1 flushleft 1 flushleft 1 flushleft 1 flushleft 1

flushleft 2 flushleft 2 flushleft 2 flushleft 2 flushleft 2 flushleft 2
flushleft 2 flushleft 2 flushleft 2 flushleft 2 flushleft 2 flushleft 2
\end{flushleft}

blub

\begin{flushleft}
flushleft single flushleft single flushleft single flushleft single
flushleft single flushleft single flushleft single flushleft single
\end{flushleft}

blablub

\begin{flushright}
flushright single flushright single flushright single flushright single
flushright single flushright single flushright single flushright single
\end{flushright}

bla

\centering
centering single centering single centering single centering single
centering single centering single centering single centering single

blabla

\raggedright
raggedright 1 raggedright 1 raggedright 1 raggedright 1 raggedright 1
raggedright 1 raggedright 1 raggedright 1 raggedright 1 raggedright 1

raggedright 2 raggedright 2 raggedright 2 raggedright 2 raggedright 2
raggedright 2 raggedright 2 raggedright 2 raggedright 2 raggedright 2

\raggedleft
raggedleft 1 raggedleft 1 raggedleft 1 raggedleft 1 raggedleft 1
raggedleft 1 raggedleft 1 raggedleft 1 raggedleft 1 raggedleft 1

raggedleft 2 raggedleft 2 raggedleft 2 raggedleft 2 raggedleft 2
raggedleft 2 raggedleft 2 raggedleft 2 raggedleft 2 raggedleft 2

%set back to justified
\raggedright{}


\subsection{Horizontal spaces}

Lines can have an hfill \hfill in the middle.
Lines can have an hfill \hspace{\fill} in the middle.
Lines can have a protected hfill \hspace*{\fill} in the middle.
Lines can have a dotted fill \dotfill in the middle.
Lines can have a rule fill \hrulefill in the middle.
Lines can have a left arrow fill \leftarrowfill in the middle.
Lines can have a right arrow fill \rightarrowfill in the middle.
Lines can have a upbrace fill \upbracefill in the middle.
Lines can have a downbrace fill \downbracefill in the middle.
Lines can have space \hspace{2cm} in the middle.
Lines can have protected space \hspace*{2cm} in the middle.

We also handle defined spaces:

Interword\ a

Visible\textvisiblespace{}a

Thin\,a

Medium\:a

Thick\;a

NegThin\negthinspace{}a

NegMed\negmedspace{}a

NegThick\negthickspace{}a

enspace\enspace{}a

enskip\enskip{}a

quad\quad{}a

qquad\qquad{}a

\subsubsection*{now some math examples:}

$a\hfill b$

$a\hspace*{2cm}b$

$a\hspace{1cm}b$

$a\hspace*{\fill} b$

$a\enskip b$

$a\enspace b$


\subsection{Vertical spaces}

Lines can have a vfill \vfill in the middle.
Lines can have a vfill \vspace{\fill} in the middle.
Lines can have a protected vfill \vspace*{\fill} in the middle.
Lines can have vertical space \vspace{2cm} in the middle.
Lines can have protected vertical space \vspace*{2cm} in the middle.

We also handle skips:

bigskip 1:\bigskip

bigskip 2:\vspace{\bigskipamount}

medskip 1:\medskip

medskip 2:\vspace{\medskipamount}

smallskip 1:\smallskip

smallskip 2:\vspace{\smallskipamount}


\end{document}
